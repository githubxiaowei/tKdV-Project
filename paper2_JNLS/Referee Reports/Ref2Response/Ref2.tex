\documentclass[11pt]{article}


\usepackage{fullpage}
\usepackage{graphicx,amsmath,amsfonts,amssymb,stmaryrd}
\usepackage{color}
\newcommand{\comment}[1]{{\color{blue} #1}}

%----------------------------- CUSTOM COMMANDS -----------------------------%

% Basic editing
\newcommand{\tocite}{{\color{blue}(to cite)}}
\newcommand{\vsp}[1]{\vspace{#1 pc} \noindent}
\newcommand{\np}{\newpage \noindent}
\newcommand{\HERE}[1]{ \vsp{#1} {\color{blue} HERE} }
% Mathy stuff
\newcommand{\abs}[1]{\lvert #1 \rvert}
\newcommand{\omavg}{\omega_0}
\newcommand{\omsig}{\sigma_{\omega}}
\newcommand{\dth}{\Delta \theta}
\newcommand{\dom}{\Delta \omega}
\newcommand{\etag}{\eta_{\gamma}}
\newcommand{\etastd}{\eta_{\text{std}}}
\newcommand{\mean}[1]{\left< #1\right>}

% Upstream and downstream notation (not used)
\newcommand{\hm}{h^-}
\newcommand{\hp}{h^+}
\newcommand{\lamm}{\lambda^-}
\newcommand{\lamp}{\lambda^+}
\newcommand{\km}{k^-}
\newcommand{\kp}{k^+}
\newcommand{\khm}{(kh)^-}
\newcommand{\khp}{(kh)^+}


%----------------------------- DOCUMENT -----------------------------%
\begin{document}

\section*{Referee 1}

% COMMENT
\noindent
\comment{A nice paper with agreement between theory and experiments, and novel new results.
}
\begin{itemize}
\item We thank the referee for these positive remarks.
\end{itemize}

% COMMENT 1
\noindent
\comment{
Use of qualitative adjectives where a quantitative description may be better:
Pg 1 Ln 39: spectacular
Pg 17 Ln 17: intriguing
Pg 18 Ln 36: modest
Pg 19 Ln 37: roughly
Pg 20 Ln 42: remarkably}

\begin{itemize}
\item We have corrected these.
\end{itemize}

% COMMENT 2
\noindent
\comment{On Pg 5 Ln 44, it is said that "the location of the peak is the same for skewness and kurtosis and appears insensitive to driving amplitude". Can you elaborate on "appears" to say whether it is or not?}

\begin{itemize}
\item Yes
\end{itemize}

% COMMENT 3
\noindent
\comment{Section 4.1 refers to Figure 3(b). If you increase N, with the transfer function become eventually collapse to one single curve? If so or indeed if not, why choose the N and Lambda in this range?}

\begin{itemize}
\item
\end{itemize}

% COMMENT 4
\noindent
\comment{For the experiments, images are taken at 60 frames/sec. The numerical differentiation of images of the free surface uses Savitzky-Golay smoothing filters. However, on Pg 18 it is mentioned that there are measurement errors and the differentiation amplifies errors. Can you comment further on this method by, say, looking at how it converges while only looking the images from 20 and 40 frames/sec or by calculating $\eta_x$ by a spectral method from eta? }

\begin{itemize}
\item
\end{itemize}

% COMMENT 5
\noindent
\comment{Fig 6 could use a label for the colourbar.}

\begin{itemize}
\item
\end{itemize}

% COMMENT 6
\noindent
\comment{Sharp steps do occur in bathymetry profiles. However, do you think Eq (53) will still apply rather well over other bathymetry profiles such as a transition with a steep finite gradient?}

\begin{itemize}
\item
\end{itemize}

% COMMENT 7
\noindent
\comment{Minor typos etc.
Pg 6 Ln 60: is leading-order -> is the leading-order
Pg 7 Ln 21: change dash to en or em dash as appropriate
Pg 16 Ln 35: Fig. 3(b) -> Fig. 3(c)
Pg 24: References [17] and [24] require capitals as they are book titles, and [27] requires a capital H for Hamiltonian}

\begin{itemize}
\item
\end{itemize}

% COMMENT 8
\noindent
\comment{}

\begin{itemize}
\item
\end{itemize}
\end{document}
