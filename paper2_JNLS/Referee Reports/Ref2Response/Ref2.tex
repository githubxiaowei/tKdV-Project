\documentclass[11pt]{article}


\usepackage{fullpage}
\usepackage{graphicx,amsmath,amsfonts,amssymb,stmaryrd}
\usepackage{color}
\newcommand{\comment}[1]{{\color{blue} #1}}

%----------------------------- CUSTOM COMMANDS -----------------------------%

% Basic editing
\newcommand{\tocite}{{\color{blue}(to cite)}}
\newcommand{\vsp}[1]{\vspace{#1 pc} \noindent}
\newcommand{\np}{\newpage \noindent}
\newcommand{\HERE}[1]{ \vsp{#1} {\color{blue} HERE} }
% Mathy stuff
\newcommand{\abs}[1]{\lvert #1 \rvert}
\newcommand{\omavg}{\omega_0}
\newcommand{\omsig}{\sigma_{\omega}}
\newcommand{\dth}{\Delta \theta}
\newcommand{\dom}{\Delta \omega}
\newcommand{\etag}{\eta_{\gamma}}
\newcommand{\etastd}{\eta_{\text{std}}}
\newcommand{\mean}[1]{\left< #1\right>}

% Upstream and downstream notation (not used)
\newcommand{\hm}{h^-}
\newcommand{\hp}{h^+}
\newcommand{\lamm}{\lambda^-}
\newcommand{\lamp}{\lambda^+}
\newcommand{\km}{k^-}
\newcommand{\kp}{k^+}
\newcommand{\khm}{(kh)^-}
\newcommand{\khp}{(kh)^+}


%----------------------------- DOCUMENT -----------------------------%
\begin{document}

\section*{Referee 1}

% COMMENT
\noindent
\comment{A nice paper with agreement between theory and experiments, and novel new results.} \\

\noindent
We thank the referee for these positive remarks. \\ \\

% COMMENT 1
\noindent
\comment{
Use of qualitative adjectives where a quantitative description may be better: \\
Pg 1 Ln 39: spectacular \\
Pg 17 Ln 17: intriguing \\
Pg 18 Ln 36: modest \\
Pg 19 Ln 37: roughly \\
Pg 20 Ln 42: remarkably} \\

\noindent
We have replaced the majority of these descriptors with more quantitative ones. In particular:
\begin{itemize}
\item The term `spectacular fashion' in the abstract was changed to `convincing evidence'.
\item The word `modest' on page 18 was replaced with `moderate'. Thanks for pointing this out -- I believe moderate was the word that was originally intended and `modest' was the result of an autocorrect.
\item The word `roughly' (appearing in several places in the original manuscript) was either replaced by `approximately' or else removed.
\item The phrase `captures the trend remarkably well' on page 20 was changed to `captures the trend closely'.
\item We have kept `intriguing' on page 17 as we believe it to be an appropriate use of the word.
\end{itemize}
\hphantom \\

% COMMENT 2
\noindent
\comment{On Pg 5 Ln 44, it is said that ``the location of the peak is the same for skewness and kurtosis and appears insensitive to driving amplitude''. Can you elaborate on "appears" to say whether it is or not?}

\begin{itemize}
\item As can bee seen in Fig. 2 (b) and (c), the location of the peak is indeed insensitive to driving amplitude to within experimental error (the level of which can be estimated by the noise in the data). Accordingly, we have replaced `appears insensitive' to simply `is insensitive'.
We had initially used the word `appears' to encourage the reader to view the figure and see for themselves. However, we now see the referee's point that this word may introduce some uncertainty in the reader's mind and have thus made the change.
\end{itemize}

% COMMENT 3
\noindent
\comment{Section 4.1 refers to Figure 3(b). If you increase $N$, with the transfer function become eventually collapse to one single curve? If so or indeed if not, why choose the $N$ and $\Lambda$ in this range?}

\begin{itemize}
\item This is a great question.... \\
Why was this value chosen? One concern is computational time; These are expensive sampling computations, considering that for each value of theta you have to run a sample.
\end{itemize}

% COMMENT 4
\noindent
\comment{For the experiments, images are taken at 60 frames/sec. The numerical differentiation of images of the free surface uses Savitzky-Golay smoothing filters. However, on Pg 18 it is mentioned that there are measurement errors and the differentiation amplifies errors. Can you comment further on this method by, say, looking at how it converges while only looking the images from 20 and 40 frames/sec or by calculating $\eta_x$ by a spectral method from eta? }

\begin{itemize}
\item
\end{itemize}

% COMMENT 5
\noindent
\comment{Fig 6 could use a label for the colourbar.}

\begin{itemize}
\item We have added  labels for the colourbars in Fig. 6.
\end{itemize}

% COMMENT 6
\noindent
\comment{Sharp steps do occur in bathymetry profiles. However, do you think Eq (53) will still apply rather well over other bathymetry profiles such as a transition with a steep finite gradient?}

\begin{itemize}
\item
\end{itemize}

% COMMENT 7
\noindent
\comment{Minor typos etc.
Pg 6 Ln 60: is leading-order -> is the leading-order
Pg 7 Ln 21: change dash to en or em dash as appropriate
Pg 16 Ln 35: Fig. 3(b) -> Fig. 3(c)
Pg 24: References [17] and [24] require capitals as they are book titles, and [27] requires a capital H for Hamiltonian}

\begin{itemize}
\item
\end{itemize}

% COMMENT 8
\noindent
\comment{}

\begin{itemize}
\item
\end{itemize}
\end{document}
