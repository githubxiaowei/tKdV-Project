\documentclass[11pt]{article}
%\documentclass[preprint, 10pt]{elsarticle}

% PACKAGES
\usepackage{graphicx, amsmath, amssymb, amsfonts, mathtools, mathrsfs, color}
\usepackage{comment, enumerate, tabularx}
\usepackage{natbib, hyperref, url}
\usepackage[margin=1in]{geometry}
%\usepackage[justification=RaggedRight]{caption}

%----------------------------------------------%
%% LATEX DEFINITIONS
%----------------------------------------------%

% Basic editing
\newcommand{\tocite}{{\color{blue}(to cite)}}
\newcommand{\vsp}[1]{\vspace{#1 pc} \noindent}
\newcommand{\np}{\newpage \noindent}
% Derivatives
\newcommand{\pd}[2]    { \frac{\partial #1} {\partial #2} }
\newcommand{\ppd}[2]  { \frac{\partial^2 #1}{{\partial #2}^2} }
\newcommand{\pdi}[2] { {\partial_#2} #1 }
\newcommand{\td}[2] { \frac{d #1} { d #2 } }
\newcommand{\grad}{\nabla}
\newcommand {\Lap} {\grad^2}
% Vectors and operators
\newcommand{\bvec}[1]{\ensuremath{\boldsymbol{#1}}}
\newcommand{\abs}[1]{\left| #1 \right|}
\newcommand{\norm}[1]{\left\| #1 \right\|}
\newcommand{\mean}[1]{\left< #1 \right>}
\newcommand{\eps}{\varepsilon}
\newcommand{\defeq}{\mathrel{\mathop:}=}

% Basic physical parameters and scales
\newcommand{\freqp}{f_p}
\newcommand{\amp}{a}
\newcommand{\ampscale}{A}
\newcommand{\etastd}{\eta_{\text std}}
% Parameters that change crossing the ADC
\newcommand{\depth}{d}
\newcommand{\dup}{\depth_{-}}
\newcommand{\ddn}{\depth_{+}}
\newcommand{\dupdn}{\depth_{\pm}}
\newcommand{\lam}{\lambda}
\newcommand{\lamup}{\lam_{-}}
\newcommand{\lamdn}{\lam_{+}}
\newcommand{\lamupdn}{\lam_{\pm}}
\newcommand{\lamfac}{N}
\newcommand{\drat}{D}
\newcommand{\dratdn}{\drat_+}
\newcommand{\dratupdn}{\drat_{\pm}}
% Statistical quantities
\newcommand{\En}{\mathcal{E}}
\newcommand{\Mo}{\mathcal{M}}
\newcommand{\skw}{\text{skew}}
\newcommand{\skwdn}{\skw_+}
\newcommand{\var}{\text{var}}
\newcommand{\varup}{\var_-}
\newcommand{\vardn}{\var_+}
% Dimensionless parameters
\newcommand{\epsup}{\eps_0}
\newcommand{\delup}{\delta_0}
% Hamiltonian and Gibbs stuff
\newcommand{\uhat}{\hat{u}}
\newcommand{\Ham}{H}
\newcommand{\Hup}{\Ham^{-}}
\newcommand{\Hdn}{\Ham^{+}}
\newcommand{\Hupdn}{\Ham^{\pm}}
\newcommand{\Gibbs}{\mathcal{G}}
\newcommand{\Gup}{\Gibbs^{-}}
\newcommand{\Gdn}{\Gibbs^{+}}
\newcommand{\Gupdn}{\Gibbs^{\pm}}
\newcommand{\thup}{\theta^{-}}
\newcommand{\thdn}{\theta^{+}}
\newcommand{\thupdn}{\theta^{\pm}}
\newcommand{\meanup}[1]{\mean{#1}_{-}}
\newcommand{\meandn}[1]{\mean{#1}_{+}}
\newcommand{\meanupdn}[1]{\mean{#1}_{\pm}}


% Dashed integral
\def\Xint#1{\mathchoice
   {\XXint\displaystyle\textstyle{#1}}%
   {\XXint\textstyle\scriptstyle{#1}}%
   {\XXint\scriptstyle\scriptscriptstyle{#1}}%
   {\XXint\scriptscriptstyle\scriptscriptstyle{#1}}%
   \!\int}
\def\XXint#1#2#3{{\setbox0=\hbox{$#1{#2#3}{\int}$}
     \vcenter{\hbox{$#2#3$}}\kern-.5\wd0}}
\def\ddashint{\Xint=}
\def\dashint{\Xint-}
\newcommand{\intt}{\dashint}%_0^{2 \pi}}
\newcommand{\dx}{\, dx}


%----------------------------------------------%
%% TITLE
%----------------------------------------------%
\begin{document}
%Deterministic and statistical truncated KdV models for anomalous waves induced by abrupt depth change
\title{The truncated KdV framework for modeling anomalous waves induced by abrupt depth changes}

\author{
C. Tyler Bolles\thanks{University of Michigan},
Andrew J.~Majda\thanks{Courant Institute of Mathematical Sciences}, 
M.~N.~J.~Moore\thanks{Florida State University}, 
Di Qi\thanks{Courant Institute of Mathematical Sciences} }
\maketitle



%----------------------------------------------------------------------------%
% Mathematical framework
%----------------------------------------------------------------------------%
\section{The truncated KdV framework}


%----------------------------------------------------------------------------%
% KdV
%----------------------------------------------------------------------------%
\subsection{The Korteweg–de Vries equation with variable depth}
Consider waves propagating unidirectionally in shallow water. Consider the surface displacement $\eta(x,t)$ and the reference frame moving with the local wave speed $\xi = x - ct$, where $c = \sqrt{g \depth}$ is the wave speed, $g$ gravity, and $\depth$ the local depth.
To first-correction in small amplitude, surface displacements are governed by the Korteweg–de Vries equation (KdV), which in dimensional form is given by
\begin{equation}
2 \eta_t + \frac{3 c}{\depth} \eta \eta_{\xi} + \frac{c \depth^2}{3} \eta_{\xi \xi \xi} = 0
\end{equation}
% The coefficients and scales can be verified on Wolfram KdV page (but what about the sign?). I would like an official source as confirmation though.

We will primarily consider the case in which waves originate from a region of constant depth, encounter an abrupt depth change, and then continue in another region of constant depth. Thus, depth will be piecewise constant
\begin{align}
\depth = 
\begin{cases}
\dup \quad \mbox{if } x<0 \\
\ddn \quad \mbox{if } x>0
\end{cases}
\end{align}
Most often, we will consider waves moving into shallower depth, so that $\dup > \ddn$. The incoming waves are randomized and generated with a peak forcing frequency of $\freqp$, which gives rise to peak wavelength $\lam = c/\freqp = \sqrt{g \depth} / \freqp$.


%----------------------------------------------------------------------------%
% Nondimensionalization
%----------------------------------------------------------------------------%
\subsection{Dimensionless variables}
\label{nondim}

% Table
%^^^^^^^^^^^^^^^^^^^^^^^^^^^^^^%
\begin{table}[h]%[htbp]
\begin{center}
\caption{Table of parameters}
\label{paramtable}
\begin{tabular}{l l l}
\hline \multicolumn{3} { c }{Parameters that are constant in a single experiment} \\
\hline Description & Notation and definition & Value in experiments \\
\hline
Peak forcing frequency		& $f_p$						& 2 Hz \\
Upstream depth			& $\dup$						& 12.5 cm \\
Downstream depth			& $\ddn$						& 3 cm (and varied) \\
Characteristic wave amplitude	& $a = \mean{\eta^2}^{1/2} $		& 0.03--0.3 cm \\
Upstream wavelength		& $\lamup = \sqrt{g \dup}/f_p$		& 56 cm \\
Amplitude-to-depth ratio		& $\epsup = a / \dup$			& 0.0024--0.024 \\
Depth-to-wavelength ratio		& $\delup = \dup / \lamup$		& 0.22 \\
Depth ratio				& $\dratdn = \ddn/\dup$			& 0.24 (and varied) \\
\hline \multicolumn{3} { c }{Parameters that change value across ADC} \\
\hline Description & Notation and definition \\
\hline
Local depth			& $\depth$			\\
Local wavespeed		& $c = \sqrt{g \depth}$	\\
Local peak wavelength	& $\lam = \sqrt{gd}/f_p$	\\
Local dimensionless depth		& $\drat = \depth/\dup$	
\end{tabular}
\end{center}
\end{table}
 %^^^^^^^^^^^^^^^^^^^^^^^^^^^^^^%
 
 \subsubsection{Most conventional nondimensionalization}
\label{nondim1}

We introduce the following characteristic scales
\begin{align}
\label{ampdefn}
&\amp = \mean{\eta^2}^{1/2} 
&&\mbox{\em characteristic wave amplitude} \\
\label{lamdefn}
&\lamupdn = \sqrt{g \dupdn} / \freqp
&&\mbox{\em characteristic wave length}
\end{align}
where $\mean{\cdot}$ indicates the mean of a quantity. Note that the characteristic wavelength $\lam = \lamupdn$ takes different values upstream and downstream of the ADC. We remark that experimental measurements indicate that the characteristic amplitude $\amp = \mean{\eta^2}^{1/2}$ is nearly the same on both values of the ADC. Hence, we will not distinguish between upstream and downstream values of $\amp$.
We introduce the following dimensionless variables
\begin{align}
&u = \eta/\amp
&&\mbox{\em dimensionless surface displacement} \\
&\tilde{x} = \frac{\pi \xi}{\lamfac \lam} = \frac{\pi (x-ct)}{\lamfac \lam}
&&\mbox{\em dimensionless position in moving frame} \\
& \tilde{t} = t \freqp / \lamfac
&&\mbox{\em dimensionless time}
\end{align}
We will consider the physical domain $\xi \in [-\lamfac \lam, \lamfac \lam]$, which is an integer multiple of the characteristic wavelength, where $\lamfac$ will be selected later (probably about 5). Above, we chosen to normalize position by the scale $L = \lamfac \lam/\pi$ so that the dimensionless domain is $\tilde{x} \in [-\pi,\pi]$. 

Accordingly, we have also chosen the time scale as $\lamfac/\freqp$, so that a wave traveling at the local wavespeed $c = \lam \freqp = \sqrt{g \depth}$ crosses the physical domain of length $2 \lamfac \lam$ in one unit of dimensionless time. Said differently, the dimensionless wave speed is $(2 \pi)^{-1}$ so that a wave crosses the domain $[-\pi,\pi]$ in unit time. Note: perhaps it is reasonable to normalize time differently, so that it is 'slow' time. Looking in Johnson, there does not seem to be a standard way to do this, it is very haphazard. So lets leave it for now. 

%----------------------------------------------------------------------------%
% COMMENT ON LAMFAC
\begin{comment}
Old Note (around May 2019): After giving it some deliberation, I believe we should use $\lamfac=1$. My reason is that the peak in the wave spectrum occurs at a frequency of $\freqp$ or wavelength of $\lam$. In the tKdV Gibbs measure, the spectrum decays monotonically, so that the peak is at the lowest resolved frequency (or largest resolved wavelength). Thus, we want $\lam$ to correspond to the largest resolved wavelength, i.e.~the length of the periodic domain in the tKdV framework.

New Thought (July 2019): Actually, it is very possible that the bandwidth (2 Hz in the experiments) sets the value of $\lamfac$ that is most appropriate, since that bandwidth sets the decay rate of the spectrum. My thinking is that our experimental forcing does not really follow the upstream Gibbs measure all that well (because we did not have that in mind), but perhaps it best approximates the tail decay of some Gibbs measure. Perhaps, the lowest frequencies (i.e. those much slower than 2 Hz that are present in the theory but almost absent in the experiments) do not affect statistics that much.
\end{comment}
%----------------------------------------------------------------------------%

Dropping the tildes, the dimensionless KdV equation takes the form
\begin{equation}
\label{dimlessKdV1}
u_t + \frac{3 \pi}{2} \epsup \drat^{-1} \, u u_x + \frac{\pi^3}{6 \lamfac^2} \delup^2 \drat \, u_{xxx} = 0
\qquad \text{for } x \in [-\pi,\pi]
\end{equation}
where dimensionless variables are as follows
\begin{align}
&\epsup = \amp/\dup \, , \\
&\delup = \dup/\lamup = \sqrt{\dup \freqp^2/g} \, \\
&\drat = {\depth}/{\dup} \, .
\end{align}
Here, $\drat = \dratupdn$ is the local dimensionless depth, which changes value crossing the ADC. In particular, since \eqref{dimlessKdV1} is in the reference frame that moves with the characteristic wave speed, we can assume that the ADC is met at $t = T_{ADC} = 0$. Then we have
\begin{equation}
\drat = 
\begin{cases}
1 		&\quad \mbox{for } t<0 \\
\dratdn 	&\quad \mbox{for } t>0
\end{cases}
\end{equation}
where $\dratdn = \ddn/\dup$.
See Table \ref{paramtable} for a summary of important parameters.

\begin{comment}
{\bf Sidenote}: In the case of constant depth, if one simply uses the most naive scales, $u = \eta/\amp$, $\tilde{x} = \xi/\lam$, $\tilde{t} = t \freqp$, then the dimensionless KdV is the more standard one:
\begin{equation}
2 u_t + 3 \eps u u_x + \frac{\delta^2}{3} u_{xxx} = 0
\end{equation}
where $\eps = a/h$ and $\delta = h/\lam$.
\end{comment}


%----------------------------------------------------------------------------%
% Johnson alternative
%----------------------------------------------------------------------------%
\subsubsection{Alternative KdV formulation following Johnson}
Section \ref{nondim} makes the implicit assumption of continuous surface displacement crossing the ADC. This is a very physically motivated assumption, but there is a small amount of debate surrounding the assumption. An alternative is to appeal to Green's law and assume the quantity $\depth^{1/4} \eta$ to be continuous across the ADC, which necessarily implies a small discontinuity in the surface (which of course cannot be true literally), as is the approach taken in Johnson't book. The dimensional KdV equation would then be expressed as
\begin{equation}
2 (\depth^{1/4} \eta)_t + \frac{3 c}{\depth^{5/4}} (\depth^{1/4} \eta) (\depth^{1/4} \eta)_{\xi} + \frac{c \depth^2}{3} (\depth^{1/4} \eta)_{\xi \xi \xi} = 0
\end{equation}
Then, the dimensionless state variable is instead 
\begin{equation}
u = \frac{\depth^{1/4} \, \eta}{ \dup^{1/4} \, \amp}
\end{equation}
and the dimensionless KdV equation takes the form
\begin{equation}
u_t + \frac{3 \pi}{\lamfac} \epsup \drat^{-5/4} \, u u_x + \frac{4 \pi^3}{3 \lamfac^3} \delup^2 \drat \, u_{xxx} = 0
\qquad \text{for } x \in [-\pi,\pi]
\end{equation}
Notice the only difference is a small change in the power of $\drat$ in the second term (changes from $\drat^{-1}$ to $\drat^{-5/4}$).
We will not use this alternative formulation.

%----------------------------------------------------------------------------%
% Energy nondimensionalization
%----------------------------------------------------------------------------%
\subsubsection{Energy nondimensionalization}

Instead of scaling the surface displacement on the characteristic amplitude, a slightly different approach is to normalize the energy to unity. Consider waves in the domain $\xi \in [-\pi L, \pi L]$ where $L$ is a length-scale we define as
\begin{equation}
L = \lamfac \lam / \pi
\end{equation}
Consider the so-called energy of $\eta$
\begin{equation}
\En [\eta] = \frac{1}{2} \int_{-\pi L}^{\pi L} \eta^2 \dx = \pi L \mean{\eta^2} \equiv E
\end{equation}
We introduce the normalized variable
\begin{align}
&u = E^{-1/2}L^{1/2} \eta \\
&\tilde{x} = \xi/L
\end{align}
Then, in dimensionless variables, we have
\begin{equation}
\En [u] = \frac{1}{2} \int_{-\pi}^{\pi} u^2 \dx = \pi \mean{u^2} = \frac{\pi L}{E} \mean{\eta^2} = 1
\end{equation}
That is, the energy is normalized to unity.

To compare to the previous formulation, we note that
\begin{align}
\amp \equiv \mean{\eta^2}^{1/2} = \sqrt{ \frac{E}{\pi L} }
\end{align}
So that the new normalization factor for $\eta$ is
\begin{equation}
E^{1/2}L^{-1/2} = \sqrt{\pi} \amp
\end{equation}
In other words, the normalizations for surface displacement used here versus that in Section \ref{nondim}
 differ by a factor of $\sqrt{\pi}$. Fortunately, the normalization for position is identical in the two. We do have to remember that the normalization factor for position $L = \lamfac \lam / \pi$ takes two different values on either side of the ADC.

Lets do the nondimensionalization generally, so that we can easily make modifications to what scales to use. In general, we introduce dimensionless variables as
\begin{align}
&u = \eta / \ampscale \\
&\tilde{x} = x / L \\
&\tilde{t} = t / T
\end{align}
where $\ampscale, L, T$ are the scales for surface displacement (amplitude), length, and time respectively, which we are free to choose later. Then the dimensionless KdV equation is
\begin{equation}
u_t + \frac{3}{2} \left( \frac{c T \ampscale}{\depth L} \right) u u_x 
+ \frac{1}{6} \left( \frac{c T \depth^2}{L^3} \right) u_{xxx} = 0
\end{equation}
where $c = \sqrt{g \depth}$.
In particular, choosing $\ampscale = \amp$, $L = \lamfac \lam/\pi$, and $T = \lamfac/\freqp$ produces \eqref{dimlessKdV1}. To see this, one has to recall $\lam = c/\freqp = \sqrt{g \depth}/\freqp$, and also has to collect all of the factors of $\depth$ that appear directly and appear in $L$.

Alternatively, we would like to choose $\ampscale = E^{1/2} L^{-1/2} = \sqrt{\pi} \amp$, $L = \lamfac \lam/ \pi$, and not sure yet about $T$. Ahh, I tried, but I cannot make sense of this because I cannot discern if $L$ in the PNAS paper and J Stat Phys paper is supposed to be dimensional or dimensionless. Also, I cannot tell of $L$ is supposed to change value crossing the ADC. The wavelength $\lam$ certainly changes value across the ADC, which suggests that $L$ should too, but on the other hand $L$ is treated as a constant throughout in the PNAS paper. So is $L$ the upstream value? I am not sure.

% Possible Flaw?
%Note: I believe this normalization is flawed because, for periodicity of $u$ to be maintained, the scale $L$ must change across the ADC. In particular, $L \sim \sqrt{\depth}$. However, the equations in our PNAS paper treat $L$ as constant throughout, even when $\drat$ changes. This seems to be a flaw to me.
% Maybe not a flow if you remember that $E^{1/2} L^{-1/2} = \sqrt{pi} \amp$ is the same on either side. Thus as L changes because wavelength changes, E changes accordingly.

%----------------------------------------------------------------------------%
%  Normalization to use
%----------------------------------------------------------------------------%
\subsubsection{Normalization to use in the paper}

We will essential use the original normalization in \eqref{dimlessKdV1}, except with the slight modification that $\eta = \sqrt{\pi} \etastd u$, so that the energy of $u$ is normalized to unity. This choice will be consistent with our PNAS paper and the {\it J Stat Phys} paper of Majda and Qi.
Thus the dimensionless variables are
\begin{align}
&u = \eta/(\sqrt{\pi} \etastd)
&&\mbox{\em dimensionless surface displacement} \\
&\tilde{x} = \frac{\pi \xi}{\lamfac \lam} = \frac{\pi (x-ct)}{\lamfac \lam}
&&\mbox{\em dimensionless position in moving frame} \\
& \tilde{t} = t \freqp / \lamfac
&&\mbox{\em dimensionless time}
\end{align}
The explanation will be something like: instead of the most conventional choice, we scale displacement on $\sqrt{\pi} \etastd$ so that energy is normalize to unity. This choice will be the most convenient for the Hamiltonian framework that we will rely on. Maybe we will have to scale time differently too to get the desired powers of $\drat$, but that will make no difference for now.
Thus, the dimensionless KdV is given by
\begin{align}
\label{dimlessKdV2}
&u_t + C_3 \drat^{-1} \, u u_x + C_2 \drat \, u_{xxx} = 0
\qquad \text{for } x \in [-\pi,\pi] \\
&C_3 = \frac{3}{2} \pi^{3/2} \epsup  \\
&C_2 = \frac{\pi^3 \delup^2}{6 \lamfac^2} 
\end{align}
When we draw a comparison to the PNAS paper, it will be something like $\epsup = E^{1/2}L^{-3/2}$ and $\delup^2 = L^{-3}$, etc.


%----------------------------------------------------------------------------%
% Hamiltonian
%----------------------------------------------------------------------------%
\subsection{Hamiltonian structure}

We now consider the Hamiltonian structure of the dimensionless variable-depth KdV equation \eqref{dimlessKdV2}, over dimensionless domain $x \in [-\pi,\pi]$. First, the spatial average of any quantity is given by
\begin{equation}
\mean{q} = \frac{1}{2\pi} \int_{-\pi}^{\pi} q \dx \, .
\end{equation}
Appealing to near Ergodicity, we will typically assume equivalence between spatial, temporal, and ensemble averages. 

We measure the state variable $u$ as the displacement of the surface from equilibrium, so that $\mean{u} = 0$. Consequently, the momentum of $u$ vanishes
\begin{equation}
\Mo[u] \defeq \int_{-\pi}^{\pi} u \dx = 0
\end{equation}
Next, the definition of the characteristic amplitude in \eqref{ampdefn} implies that $\mean{u^2} = \pi^{-1}$. Consequently, the energy of $u$ is fixed as
\begin{equation}
\En[u] \defeq \frac{1}{2} \int_{-\pi}^{\pi} u^2 \dx = 1
\end{equation}

We introduce the Fourier series
\begin{equation}
u = \sum_{k=-\Lambda}^{\Lambda} \uhat_k e^{i k x}
\end{equation}
where
\begin{equation}
\uhat_k = \frac{1}{2 \pi} \int_{-\pi}^{\pi} u(x) e^{-i k x} \dx
\end{equation}
The above sum can be infinite $\Lambda = \infty$ or truncated at a finite wavenumber $\Lambda < \infty$.
Further, since $u$ is real valued, we have $\uhat_{-k} = {\uhat_{k}}^*$.
By Parseval's identity, we have
\begin{equation}
\En[u] = 1 = \frac{1}{2} \int u^2 \dx = \pi \sum_{k=-\Lambda}^{\Lambda} \abs{\uhat_k}^2 = 2 \pi \sum_{k=1}^{\Lambda} \abs{\uhat_k}^2
\end{equation}
% The above agrees with the first line on top of p. 4 in Majda and Qi J Stat Phys 2019.

We introduce the cubic and quadratic Hamiltonian components as
\begin{align}
H_3 = \frac{1}{6} \int_{-\pi}^{\pi} u^3 \dx	\, , \qquad
H_2 = \frac{1}{2} \int_{-\pi}^{\pi} u_x^2 \dx	\, .
\end{align}

Then the Hamiltonian of \eqref{dimlessKdV1} is given by
\begin{equation}
\Ham = C_3 \drat^{-1} \, H_3 - C_2 \drat \, H_2
\end{equation}
where $\drat$ changes value across the ADC. More specifically, there exists an upstream and a downstream Hamiltonian
\begin{align}
&\Hup = C_3 \, H_3 - C_2 \, H_2 \\
&\Hdn = C_3 \dratdn^{-1} \, H_3 - C_2 \dratdn \, H_2
\end{align}
$\Hup$ is a conserved quantity for $t<0$ and $\Hdn$ is a conserved quantity for $t>0$. Each Hamiltonian $\Ham^{\pm}$ generates a corresponding Gibbs measure $\Gupdn$ 
\begin{align}
\Gupdn = C_{\thupdn} \exp(-\thupdn \Hupdn) \delta(\En - \pi)
\end{align}
where $\theta = \thupdn$ is the inverse temperature and $C_{\theta}$ a constant (the partition function) that depends on $\theta$. Each measure has corresponding ensemble mean $\mean{}_{\pm}$. 


%----------------------------------------------------------------------------%
% Matching
%----------------------------------------------------------------------------%
\subsection{Matching condition}

The statistical matching condition is
\begin{equation}
\meanup{\Hdn} = \meandn{\Hdn}
\end{equation}
An experimental observation is that the incoming skewness is small, $\meanup{H_3} \approx 0$. Inserting and simplifying gives
\begin{equation}
\label{H3H2}
\frac{\meandn{H_3^+}} {\meandn{H_2^+} - \meanup{H_2^+}} = \frac{C_2}{C_3} \dratdn^2
\end{equation}
%where $\Delta \mean{H_2} =  \meandn{H_2} - \meanup{H_2}$  indicates the difference between the upstream and downstream values.
To convert to dimensional, i.e.~experimental, values we first note that
\begin{align}
\mean{H_3^+} = \frac{\pi}{3} \mean{u^3} = 
\frac{1}{3} \pi^{-1/2} \frac{\mean{\eta^3}}{\etastd^3} = \frac{1}{3} \pi^{-1/2} \skw(\eta)
\end{align}
Second, to convert the downstream surface slope, we note
\begin{align}
\pd{u}{\tilde{x}} = \frac{\lamfac \lamdn}{\pi^{3/2} \etastd} \pd{\eta}{\xi}
\end{align}
where slope with respect to $\xi$ is the same as with respect to dimensional $x$.
Note: I believe it is important here to use $\lamdn$ since we are matching $\Hdn$.
Hence, we have
\begin{align}
&\mean{H_2^+} = \pi \mean{ \left( \pd{u}{\tilde{x}} \right)^2} = 
\left( \frac{\lamfac \lamdn}{\pi \etastd} \right)^2 \var(\eta_x)
\end{align}
%
Then algebra gives
\begin{equation}
\frac{\meandn{H_3}} {\meandn{H_2} - \meanup{H_2}} = 
\frac{\pi^{3/2} \, \etastd^2} {3 \lamfac^2 \lamdn^2} 
\left( \frac{\skwdn(\eta)} { \vardn(\eta_x) - \varup(\eta_x)} \right)
\end{equation}
Also we have the simplification
\begin{equation}
\frac{C_2}{C_3} = \frac{\pi^{3/2} \delup^2}{9 \lamfac^2 \epsup}
\end{equation}
Hence, \eqref{H3H2} gives
\begin{equation}
\frac{\skwdn(\eta)} {\vardn(\eta_x) - \varup(\eta_x)} = \frac{1}{3} \epsup^{-3} \dratdn^3
\end{equation}
In particular, the ratio on the left, which can be measured in the experiments, is predicted to scale as the inverse cube of the wave amplitude and as the square of the depth ratio.
% Note: previously I had D^2 on the RHS. The power of 3 comes from using lamdn consistently in the nondimensionalization.

\subsection{Derivation with $\lam$ changing across ADC}
This derivation must be faulty, but here it is so that I remember it.
Aha, I think the error is as follows: The matching condition is for $\Hdn$ only, using the two measures. So you have to use the downstream scales, in particular $\lamdn$. 

\vsp{1}
Anyway, here is the flawed derivation, so that we remember it.
Algebra gives
\begin{equation}
\frac{\meandn{H_3}} {\meandn{H_2} - \meanup{H_2}} = 
\frac{\pi^{3/2} \, \etastd^2} {3 \lamfac^2} 
\left( \frac{\skwdn(\eta)}{ \lamdn^2 \vardn(\eta_x) - \lamup^2 \varup(\eta_x)} \right) = 
\frac{\pi^{3/2} \, \etastd^2} {3 \lamfac^2 \lamup^2} 
\left( \frac{\skwdn(\eta)} {\dratdn \vardn(\eta_x) - \varup(\eta_x)} \right)
\end{equation}
where in the second equality, we have used the fact that
\begin{equation}
\frac{\lamdn^2}{\lamup^2} = \frac{\ddn}{\dup} = \dratdn
\end{equation}
Also we have the simplification
\begin{equation}
\frac{C_2}{C_3} = \frac{\pi^{3/2} \delup^2}{9 \lamfac^2 \epsup}
\end{equation}
Hence, \eqref{H3H2} gives
\begin{equation}
\frac{\skwdn(\eta)} {\dratdn \vardn(\eta_x) - \varup(\eta_x)} = \frac{1}{3} \epsup^{-3} \dratdn^2
\end{equation}
In particular, the ratio on the left, which can be measured in the experiments, is predicted to scale as the inverse cube of the wave amplitude and as the square of the depth ratio.

%^^^^^^^^^^^^^^^^^^^^^^^^^^^^^^%
\begin{figure}%[htbp]
\begin{center}
\includegraphics[width = 0.80 \textwidth]{../Figs/SkewAmp.pdf}
\caption{\label{fig1} 
The ratio ${\skw(\eta)}/{\Delta \var(\eta_x)}$ plots against dimensionless wave amplitude.
}
\end{center}
\end{figure}
 %^^^^^^^^^^^^^^^^^^^^^^^^^^^^^^%



%----------------------------------------------------------%
% Comparison with experiments
%----------------------------------------------------------%
\section{Comparison with experiments}

\subsection{Comparison of basic features}

\subsection{New experimental measurements guided by theory}

%\bibliographystyle{plain}
%\bibliography{Notesbib}

\end{document}
