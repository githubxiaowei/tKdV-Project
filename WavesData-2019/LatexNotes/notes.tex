\documentclass[11pt]{article}
%\documentclass[preprint, 10pt]{elsarticle}

% PACKAGES
\usepackage{graphicx, amsmath, amssymb, amsfonts, mathtools, mathrsfs, color}
\usepackage{comment, enumerate, tabularx}
\usepackage{natbib, hyperref, url}
\usepackage[margin=1in]{geometry}
%\usepackage[justification=RaggedRight]{caption}

%----------------------------------------------%
%% LATEX DEFINITIONS
%----------------------------------------------%

% Basic editing
\newcommand{\tocite}{{\color{blue}(to cite)}}
\newcommand{\vsp}[1]{\vspace{#1 pc} \noindent}
\newcommand{\np}{\newpage \noindent}
% Derivatives
\newcommand{\pd}[2]    { \frac{\partial #1} {\partial #2} }
\newcommand{\ppd}[2]  { \frac{\partial^2 #1}{{\partial #2}^2} }
\newcommand{\pdi}[2] { {\partial_#2} #1 }
\newcommand{\td}[2] { \frac{d #1} { d #2 } }
\newcommand{\grad}{\nabla}
\newcommand {\Lap} {\grad^2}
% Vectors and operators
\newcommand{\bvec}[1]{\ensuremath{\boldsymbol{#1}}}
\newcommand{\abs}[1]{\left| #1 \right|}
\newcommand{\norm}[1]{\left\| #1 \right\|}
\newcommand{\mean}[1]{\left< #1 \right>}
\newcommand{\eps}{\varepsilon}

% Experimental parameters
\newcommand{\lamm}{\lambda^{-}}
\newcommand{\lamp}{\lambda^{+}}
\newcommand{\hm}{h^{-}}
\newcommand{\hp}{h^{+}}
\newcommand{\etastd}{\eta_{\text std}}
% Theory for truncated KdV and stat mech
\newcommand{\uhat}{\hat{u}}
\newcommand{\RR}{\mathbb{R}}
\newcommand{\Real}{\text{Re}}
\newcommand{\Fspace}{\mathscr{F}_{\Lambda}}
\newcommand{\Proj}{P_{\Lambda}}
\newcommand{\sumk}{\sum_{k=1}^{\Lambda}}
\newcommand{\sumn}{\sum_{n=1}^{N}}
\newcommand{\uDir}{u_{\text{Dir}}}
\newcommand{\Gibbs}{\mathcal{G}}
\newcommand{\ampp}{\mathcal{A}}
\newcommand{\lamfac}{N}

% New
\newcommand{\Hp}{H^{+}}
\newcommand{\Hm}{H^{-}}
\newcommand{\Gp}{\Gibbs^{+}}
\newcommand{\Gm}{\Gibbs^{-}}

% Dashed integral
\def\Xint#1{\mathchoice
   {\XXint\displaystyle\textstyle{#1}}%
   {\XXint\textstyle\scriptstyle{#1}}%
   {\XXint\scriptstyle\scriptscriptstyle{#1}}%
   {\XXint\scriptscriptstyle\scriptscriptstyle{#1}}%
   \!\int}
\def\XXint#1#2#3{{\setbox0=\hbox{$#1{#2#3}{\int}$}
     \vcenter{\hbox{$#2#3$}}\kern-.5\wd0}}
\def\ddashint{\Xint=}
\def\dashint{\Xint-}
\newcommand{\intt}{\dashint}%_0^{2 \pi}}
\newcommand{\dx}{\, dx}


%----------------------------------------------%
%% TITLE
%----------------------------------------------%
\begin{document}
%Deterministic and statistical truncated KdV models for anomalous waves induced by abrupt depth change
\title{The truncated KdV framework for modeling anomalous waves induced by abrupt depth changes}

%\author{
% Tyler Bolles
%Andrew J.~Majda\thanks{Courant Institute of Mathematical Sciences}, 
%M.~N.~J.~Moore\thanks{Florida State University}, 
%Di Qi\thanks{Courant Institute of Mathematical Sciences} }
\maketitle



%----------------------------------------------------------%
% Mathematical framework
%----------------------------------------------------------%
\section{The truncated KdV framework}

\subsection{KdV with depth change}
In dimensional form the Korteweg–de Vries equation is given by
\begin{equation}
2 \eta_t + \frac{3 c}{h_0} \eta \eta_{\xi} + \frac{c h_0^2}{3} \eta_{\xi \xi \xi} = 0
\end{equation}
 
\subsection{Nondimensionalization}

The dimensionless KdV equation is
\begin{align}
\label{varKdV}
u_t + \ampp D^{-3/2} \, u u_x + \mu D^{1/2} \, u_{xxx} = 0
\end{align}
with the dimensionless parameters
\begin{align}
\ampp = \eps \delta^{-2} = \frac{ag}{h_0^2 f_p^2} , \qquad
\mu = \frac{4 \pi^2}{9 \lamfac^2}
\end{align}
Here, $\lamfac$ is the number of peak wavelengths in the periodic domain (perhaps we should set $\lamfac=1$, see note below). Above, the displacement $u$ has been scaled on the characteristic wave amplitude $a = \mean{\eta^2}^{1/2}$, and position $x$ has been scaled on the upstream characteristic wavelength $\lambda_0 = \sqrt{g h_0}/f_p$.
Then the Hamiltonian is given by
\begin{align}
\label{Hamiltonian}
& \Hm = \ampp H_3 - \mu H_2 \\
& \Hp = \ampp D_0^{-3/2} H_3 - \mu D_0^{1/2} H_2
\end{align}
where
\begin{align}
&H_3 = \intt u^3 \dx	\, , \\
&H_2 = \intt u_x^2 \dx	\, .
\end{align}

Note: after giving it some deliberation, I believe we should use $\lamfac=1$. My reason is that the peak in the wave spectrum occurs at a frequency of $f_p$ or wavelength of $\lambda_p$. In the tKdV Gibbs measure, the spectrum decays monotonically, so that the peak is at the lowest resolved frequency (or largest resolved wavelength). Thus, we want $\lambda_p$ to correspond to the largest resolved wavelength, i.e.~the length of the periodic domain in the tKdV framework.

%Remark: We note that another possibility is to match the quantity $D^{1/4} \eta$ at the abrupt depth change as would be consistent with the analysis of Johnson. If we were to do this, we would obtain only slightly different powers of $D$ in the coefficients: $D^{-7/4}$ and $D^{1/2}$ respectively.

% Table
%^^^^^^^^^^^^^^^^^^^^^^^^^^^^^^%
\begin{table}[h]%[htbp]
\begin{center}
\caption{Table of parameters}
\label{paramtable}
\begin{tabular}{l l l l}
\hline \multicolumn{4} { c }{Parameters that are constant throughout the domain} \\
\hline Description & Notation and definition & Units & Value in experiments \\
\hline
Peak forcing frequency		& $f_p$						& frequency	& 2 Hz \\
Characteristic wave amplitude	& $a = \mean{\eta^2}^{1/2}$		& length		& 0.03--0.3 cm \\
Upstream wavelength			& $\lambda_0 = \sqrt{g h_0}/f_p$	& length		& 56 cm \\
Amplitude-to-depth ratio		& $\eps = a / h_0$				& dimensionless	& 0.0024--0.024 \\
Depth-to-wavelength ratio		& $\delta = h_0 / \lambda_0$		& dimensionless		& 0.22 \\
Amplitude parameter			& $\ampp = \eps \delta^{-2} = ag h_0^{-2} f_p^{-2}$	
& dimensionless		& 0.05--0.5\\
Catch-all parameter		& $\mu = 4 \pi^2 / (9 \lamfac^2)$			& dimensionless		& 0.27 if $\lamfac=4$ \\
\hline \multicolumn{4} { c }{Parameters that change value across ADC} \\
\hline Description & Notation and definition & Units \\
\hline
Local depth			& $d$					& length \\
Local wavespeed		& $c = \sqrt{gd}$			& speed \\
Local peak wavelength	& $\lambda_p = \sqrt{gd}/f_p$	& length \\
Characteristic wavelength	& $\lambda_c = \lamfac \lambda_p$	& length \\
Depth ratio			& $D = d/h_0$				& dimensionless
\end{tabular}
\end{center}
\end{table}
 %^^^^^^^^^^^^^^^^^^^^^^^^^^^^^^%

\subsection{Matching condition}

The statistical matching condition is
\begin{equation}
\mean{\Hp}_{\Gm} = \mean{\Hp}_{\Gp}
\end{equation}


%----------------------------------------------------------%
% Comparison with experiments
%----------------------------------------------------------%
\section{Comparison with experiments}

\subsection{Comparison of basic features}

\subsection{New experimental measurements guided by theory}

%\bibliographystyle{plain}
%\bibliography{Notesbib}

\end{document}
