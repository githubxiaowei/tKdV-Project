\documentclass[11pt]{article}
%\documentclass[preprint, 10pt]{elsarticle}

% PACKAGES
\usepackage{graphicx, amsmath, amssymb, amsfonts, mathtools, mathrsfs, color}
\usepackage{comment, enumerate, tabularx}
\usepackage{natbib, hyperref, url}
\usepackage[margin=1in]{geometry}
%\usepackage[justification=RaggedRight]{caption}

%----------------------------------------------%
%% LATEX DEFINITIONS
%----------------------------------------------%

% Basic editing
\newcommand{\tocite}{{\color{blue}(to cite)}}
\newcommand{\vsp}[1]{\vspace{#1 pc} \noindent}
\newcommand{\np}{\newpage \noindent}
% Derivatives
\newcommand{\pd}[2]    { \frac{\partial #1} {\partial #2} }
\newcommand{\ppd}[2]  { \frac{\partial^2 #1}{{\partial #2}^2} }
\newcommand{\pdi}[2] { {\partial_#2} #1 }
\newcommand{\td}[2] { \frac{d #1} { d #2 } }
\newcommand{\grad}{\nabla}
\newcommand {\Lap} {\grad^2}
% Vectors and operators
\newcommand{\bvec}[1]{\ensuremath{\boldsymbol{#1}}}
\newcommand{\abs}[1]{\left| #1 \right|}
\newcommand{\norm}[1]{\left\| #1 \right\|}
\newcommand{\mean}[1]{\left< #1 \right>}
\newcommand{\eps}{\varepsilon}

% Basic physical parameters and scales
\newcommand{\depth}{h}
\newcommand{\dup}{\depth^{-}}
\newcommand{\ddn}{\depth^{+}}
\newcommand{\freqp}{f_p}
\newcommand{\lam}{\lambda}
\newcommand{\lamup}{\lam^{-}}
\newcommand{\lamfac}{N}
% More parameters
\newcommand{\etastd}{\eta_{\text std}}
\newcommand{\amp}{a}
% Statistical quantities
\newcommand{\skw}{\text{skew}}
\newcommand{\var}{\text{var}}
% Dimensionless parameters
\newcommand{\epsup}{\eps_0}
\newcommand{\delup}{\delta_0}
\newcommand{\drat}{D}
\newcommand{\ampp}{\mathcal{A}}
% Hamiltonian and Gibbs stuff
\newcommand{\Hp}{H^{+}}
\newcommand{\Hm}{H^{-}}
\newcommand{\Gibbs}{\mathcal{G}}
\newcommand{\Gp}{\Gibbs^{+}}
\newcommand{\Gm}{\Gibbs^{-}}
\newcommand{\meanup}[1]{\mean{#1}_{-}}
\newcommand{\meandn}[1]{\mean{#1}_{+}}



% Dashed integral
\def\Xint#1{\mathchoice
   {\XXint\displaystyle\textstyle{#1}}%
   {\XXint\textstyle\scriptstyle{#1}}%
   {\XXint\scriptstyle\scriptscriptstyle{#1}}%
   {\XXint\scriptscriptstyle\scriptscriptstyle{#1}}%
   \!\int}
\def\XXint#1#2#3{{\setbox0=\hbox{$#1{#2#3}{\int}$}
     \vcenter{\hbox{$#2#3$}}\kern-.5\wd0}}
\def\ddashint{\Xint=}
\def\dashint{\Xint-}
\newcommand{\intt}{\dashint}%_0^{2 \pi}}
\newcommand{\dx}{\, dx}


%----------------------------------------------%
%% TITLE
%----------------------------------------------%
\begin{document}
%Deterministic and statistical truncated KdV models for anomalous waves induced by abrupt depth change
\title{The truncated KdV framework for modeling anomalous waves induced by abrupt depth changes}

%\author{
% Tyler Bolles
%Andrew J.~Majda\thanks{Courant Institute of Mathematical Sciences}, 
%M.~N.~J.~Moore\thanks{Florida State University}, 
%Di Qi\thanks{Courant Institute of Mathematical Sciences} }
\maketitle



%----------------------------------------------------------%
% Mathematical framework
%----------------------------------------------------------%
\section{The truncated KdV framework}

\subsection{The Korteweg–de Vries equation with variable depth}
Consider waves propagating unidirectionally in shallow water. Consider the surface displacement $\eta(x,t)$ and the reference frame moving with the local wave speed $\xi = x - ct$, where $c = \sqrt{g \depth}$ is the wave speed, $g$ gravity, and $\depth$ the local depth.
To first-correction in small amplitude, surface displacements are governed by the Korteweg–de Vries equation (KdV), which in dimensional form is given by
\begin{equation}
2 \eta_t + \frac{3 c}{\depth} \eta \eta_{\xi} + \frac{c \depth^2}{3} \eta_{\xi \xi \xi} = 0
\end{equation}
% The coefficients and scales can be verified on Wolfram KdV page (but what about the sign?). I would like an official source as confirmation though.

We will primarily consider the case in which waves originate from a region of constant depth, encounter an abrupt depth change, and then continue in another region of constant depth. Thus, depth will be piecewise constant
\begin{align}
\depth = 
\begin{cases}
\it \dup \quad \mbox{if } x<0 \\
\it \ddn \quad \mbox{if } x>0
\end{cases}
\end{align}
Most often, we will consider waves moving into shallower depth, so that $\ddn < \dup$. The incoming waves are randomized and generated with a peak forcing frequency of $\freqp$. 

\subsection{Dimensionless variables}

We introduce the following characteristic scales
\begin{align}
&\amp = \mean{\eta^2}^{1/2} 
&&\mbox{\em characteristic wave amplitude} \\
&\lam = \sqrt{g \depth} / \freqp
&&\mbox{\em characteristic wave length}
\end{align}
where $\mean{\cdot}$ indicates the mean of a quantity. Note that the characteristic wavelength $\lam = \lam(x)$ takes different values upstream and downstream of the ADC.
We introduce the following dimensionless variables
\begin{align}
&u = \eta/\amp
&&\mbox{\em dimensionless surface displacement} \\
&\tilde{x} = \frac{2 \pi \xi}{\lamfac \lam} = \frac{2 \pi (x-ct)}{\lamfac \lam}
&&\mbox{\em dimensionless position in moving frame} \\
& \tilde{t} = t \freqp
&&\mbox{\em dimensionless time} \\
\end{align}
Then the dimensionless KdV becomes
\begin{equation}
u_t + \frac{3 \pi}{\lamfac} \epsup \drat^{-1} u u_x + \frac{4 \pi^3}{3 \lamfac^3} \delup^2 \drat u_{xxx} = 0
\end{equation}
where $\epsup = \amp/\dup$, $\delup = \dup/\lamup = \sqrt{\dup \freqp^2/g}$, and $\drat = {\depth}/{\dup}$ is the local dimensionless depth.

\vsp{2}
Sidenote: If one simply uses the most naive scales, $u = \eta/\amp$, $\tilde{x} = \xi/\lam$, $\tilde{t} = t \freqp$, then the dimensionless KdV is the more standard one:
\begin{equation}
2 u_t + 3 \eps u u_x + \frac{\delta^2}{3} u_{xxx} = 0
\end{equation}
where $\eps = a/h$ and $\delta = h/\lam$.


\vsp{10}
Old notes:
The dimensionless KdV equation is
\begin{align}
\label{varKdV}
u_t + \ampp D^{-3/2} \, u u_x + \mu D^{1/2} \, u_{xxx} = 0
\end{align}
with the dimensionless parameters
\begin{align}
\label{ndKdV}
\ampp = \eps \delta^{-2} = \frac{ag}{h_0^2 f_p^2} , \qquad
\mu = \frac{4 \pi^2}{9 \lamfac^2}
\end{align}
Here, the displacement has been scaled on the characteristic wave amplitude, $u = \eta/a$, where $a = \etastd= \mean{\eta^2}^{1/2}$.
As a consequence, $u$ has fixed energy
\begin{equation}
E[u] = \frac{1}{2} \intt u^2 \dx = \frac{1}{2}
\end{equation}

In \eqref{ndKdV}, position $x$ has been scaled on the upstream characteristic wavelength $\lambda_0 = \sqrt{g h_0}/f_p$. Further, $\lamfac$ is the number of peak wavelengths in the periodic domain (perhaps we should set $\lamfac=1$, see note below).
Then the Hamiltonian is given by
\begin{align}
\label{Hamiltonian}
& \Hm = \ampp H_3 - \mu H_2 \\
& \Hp = \ampp D^{-3/2} H_3 - \mu D^{1/2} H_2
\end{align}
where
\begin{align}
&H_3 = \frac{1}{6} \intt u^3 \dx	\, , \\
&H_2 = \frac{1}{2} \intt u_x^2 \dx	\, .
\end{align}

Note: after giving it some deliberation, I believe we should use $\lamfac=1$. My reason is that the peak in the wave spectrum occurs at a frequency of $f_p$ or wavelength of $\lambda_p$. In the tKdV Gibbs measure, the spectrum decays monotonically, so that the peak is at the lowest resolved frequency (or largest resolved wavelength). Thus, we want $\lambda_p$ to correspond to the largest resolved wavelength, i.e.~the length of the periodic domain in the tKdV framework.

%Remark: We note that another possibility is to match the quantity $D^{1/4} \eta$ at the abrupt depth change as would be consistent with the analysis of Johnson. If we were to do this, we would obtain only slightly different powers of $D$ in the coefficients: $D^{-7/4}$ and $D^{1/2}$ respectively.

NOTE: We need to include the bandwidth, 2 Hz, in the theory somehow. For example, the peak forcing frequency and bandwidth together can be used to estimate the decay in the spectrum.

% Table
%^^^^^^^^^^^^^^^^^^^^^^^^^^^^^^%
\begin{table}[h]%[htbp]
\begin{center}
\caption{Table of parameters}
\label{paramtable}
\begin{tabular}{l l l l}
\hline \multicolumn{4} { c }{Parameters that are constant throughout the domain} \\
\hline Description & Notation and definition & Units & Value in experiments \\
\hline
Peak forcing frequency		& $f_p$						& frequency	& 2 Hz \\
Characteristic wave amplitude	& $a = \mean{\eta^2}^{1/2} = \etastd$		& length		& 0.03--0.3 cm \\
Upstream wavelength			& $\lambda_0 = \sqrt{g h_0}/f_p$	& length		& 56 cm \\
Amplitude-to-depth ratio		& $\eps = a / h_0$				& dimensionless	& 0.0024--0.024 \\
Depth-to-wavelength ratio		& $\delta = h_0 / \lambda_0$		& dimensionless		& 0.22 \\
Amplitude parameter			& $\ampp = \eps \delta^{-2} = a \lambda_0^2 h_0^{-3} = ag h_0^{-2} f_p^{-2}$	
& dimensionless		& 0.05--0.5\\
Catch-all parameter		& $\mu = 4 \pi^2 / (9 \lamfac^2)$			& dimensionless		& 0.27 if $\lamfac=4$ \\
\hline \multicolumn{4} { c }{Parameters that change value across ADC} \\
\hline Description & Notation and definition & Units \\
\hline
Local depth			& $d$					& length \\
Local wavespeed		& $c = \sqrt{gd}$			& speed \\
Local peak wavelength	& $\lambda_p = \sqrt{gd}/f_p$	& length \\
Characteristic wavelength	& $\lambda_c = \lamfac \lambda_p$	& length \\
Depth ratio			& $D = d/h_0$				& dimensionless
\end{tabular}
\end{center}
\end{table}
 %^^^^^^^^^^^^^^^^^^^^^^^^^^^^^^%

\subsection{Matching condition}

The statistical matching condition is
\begin{equation}
\meanup{\Hp} = \meandn{\Hp}
\end{equation}
An experimental observation is that the incoming skewness is small, thus
\begin{equation}
\meanup{H_3} \approx 0
\end{equation}
which immediately gives
\begin{equation}
\ampp D^{-3/2} \meandn{H_3} = \mu D^{1/2} \left( \meandn{H_2} - \meanup{H_2} \right)
\end{equation}
or equivalently
\begin{equation}
\label{H3H2}
\meandn{H_3} = \frac{\mu D^{2}}{\ampp} \, \Delta \mean{H_2}
\end{equation}
where $\Delta \mean{H_2} =  \meandn{H_2} - \meanup{H_2}$  indicates the difference between the upstream and downstream values.

To convert to dimensional, i.e.~experimental, values we use
\begin{align}
& \intt u^3 \dx = \frac{\mean{\eta^3}}{a^3} = 
\frac{\mean{\eta^3}}{\mean{\eta^2}^{3/2}} = \skw(\eta) \\
& \intt u_x^2 \dx = \left(\frac{\lambda_0}{a} \right)^2 \mean{\eta_x^2} 
= \left(\frac{\lambda_0}{a} \right)^2 \var(\eta_x)
\end{align}
Then \eqref{H3H2} gives the relationship
\begin{equation}
\frac{\skw(\eta)}{\Delta \var(\eta_x)} =
3 \mu D^2 \left( \frac{h_0}{a} \right)^3 = 3 \eps^{-3} \mu D^2
\end{equation}
In particular, the ratio on the left, which can be measured in the experiments, is predicted to scale as the inverse cube of the wave amplitude (as well as the square of the depth ratio).

%^^^^^^^^^^^^^^^^^^^^^^^^^^^^^^%
\begin{figure}%[htbp]
\begin{center}
\includegraphics[width = 0.80 \textwidth]{../Figs/SkewAmp.pdf}
\caption{\label{fig1} 
The ratio ${\skw(\eta)}/{\Delta \var(\eta_x)}$ plots against dimensionless wave amplitude.
}
\end{center}
\end{figure}
 %^^^^^^^^^^^^^^^^^^^^^^^^^^^^^^%


%----------------------------------------------------------%
% Comparison with experiments
%----------------------------------------------------------%
\section{Comparison with experiments}

\subsection{Comparison of basic features}

\subsection{New experimental measurements guided by theory}

%\bibliographystyle{plain}
%\bibliography{Notesbib}

\end{document}
