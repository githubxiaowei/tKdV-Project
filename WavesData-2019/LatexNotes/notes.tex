\documentclass[11pt]{article}
%\documentclass[preprint, 10pt]{elsarticle}

% PACKAGES
\usepackage{graphicx, amsmath, amssymb, amsfonts, mathtools, mathrsfs, color}
\usepackage{comment, enumerate, tabularx}
\usepackage{natbib, hyperref, url}
\usepackage[margin=1in]{geometry}
%\usepackage[justification=RaggedRight]{caption}

%----------------------------------------------%
%% LATEX DEFINITIONS
%----------------------------------------------%

% Basic editing
\newcommand{\tocite}{{\color{blue}(to cite)}}
\newcommand{\vsp}[1]{\vspace{#1 pc} \noindent}
\newcommand{\np}{\newpage \noindent}
% Derivatives
\newcommand{\pd}[2]    { \frac{\partial #1} {\partial #2} }
\newcommand{\ppd}[2]  { \frac{\partial^2 #1}{{\partial #2}^2} }
\newcommand{\pdi}[2] { {\partial_#2} #1 }
\newcommand{\td}[2] { \frac{d #1} { d #2 } }
\newcommand{\grad}{\nabla}
\newcommand {\Lap} {\grad^2}
% Vectors and operators
\newcommand{\bvec}[1]{\ensuremath{\boldsymbol{#1}}}
\newcommand{\abs}[1]{\left| #1 \right|}
\newcommand{\norm}[1]{\left\| #1 \right\|}
\newcommand{\mean}[1]{\left< #1 \right>}
\newcommand{\eps}{\varepsilon}
\newcommand{\defeq}{\mathrel{\mathop:}=}

% Basic physical parameters and scales
\newcommand{\depth}{h}
\newcommand{\dup}{\depth_{-}}
\newcommand{\ddn}{\depth_{+}}
\newcommand{\freqp}{f_p}
\newcommand{\lam}{\lambda}
\newcommand{\lamup}{\lam_{-}}
\newcommand{\lamfac}{N}
\newcommand{\amp}{a}
%\newcommand{\etastd}{\eta_{\text std}}
% Statistical quantities
\newcommand{\skw}{\text{skew}}
\newcommand{\var}{\text{var}}
% Dimensionless parameters
\newcommand{\epsup}{\eps_0}
\newcommand{\delup}{\delta_0}
\newcommand{\drat}{D}
\newcommand{\dratdn}{\drat_0}
\newcommand{\ampp}{\mathcal{A}}
% Hamiltonian and Gibbs stuff
\newcommand{\Ham}{H}
\newcommand{\Hup}{\Ham^{-}}
\newcommand{\Hdn}{\Ham^{+}}
\newcommand{\Gibbs}{\mathcal{G}}
\newcommand{\Gup}{\Gibbs^{-}}
\newcommand{\Gdn}{\Gibbs^{+}}
\newcommand{\meanup}[1]{\mean{#1}_{-}}
\newcommand{\meandn}[1]{\mean{#1}_{+}}



% Dashed integral
\def\Xint#1{\mathchoice
   {\XXint\displaystyle\textstyle{#1}}%
   {\XXint\textstyle\scriptstyle{#1}}%
   {\XXint\scriptstyle\scriptscriptstyle{#1}}%
   {\XXint\scriptscriptstyle\scriptscriptstyle{#1}}%
   \!\int}
\def\XXint#1#2#3{{\setbox0=\hbox{$#1{#2#3}{\int}$}
     \vcenter{\hbox{$#2#3$}}\kern-.5\wd0}}
\def\ddashint{\Xint=}
\def\dashint{\Xint-}
\newcommand{\intt}{\dashint}%_0^{2 \pi}}
\newcommand{\dx}{\, dx}


%----------------------------------------------%
%% TITLE
%----------------------------------------------%
\begin{document}
%Deterministic and statistical truncated KdV models for anomalous waves induced by abrupt depth change
\title{The truncated KdV framework for modeling anomalous waves induced by abrupt depth changes}

\author{
C. Tyler Bolles\thanks{University of Michigan},
Andrew J.~Majda\thanks{Courant Institute of Mathematical Sciences}, 
M.~N.~J.~Moore\thanks{Florida State University}, 
Di Qi\thanks{Courant Institute of Mathematical Sciences} }
\maketitle



%----------------------------------------------------------%
% Mathematical framework
%----------------------------------------------------------%
\section{The truncated KdV framework}


\subsection{The Korteweg–de Vries equation with variable depth}
Consider waves propagating unidirectionally in shallow water. Consider the surface displacement $\eta(x,t)$ and the reference frame moving with the local wave speed $\xi = x - ct$, where $c = \sqrt{g \depth}$ is the wave speed, $g$ gravity, and $\depth$ the local depth.
To first-correction in small amplitude, surface displacements are governed by the Korteweg–de Vries equation (KdV), which in dimensional form is given by
\begin{equation}
2 \eta_t + \frac{3 c}{\depth} \eta \eta_{\xi} + \frac{c \depth^2}{3} \eta_{\xi \xi \xi} = 0
\end{equation}
% The coefficients and scales can be verified on Wolfram KdV page (but what about the sign?). I would like an official source as confirmation though.

We will primarily consider the case in which waves originate from a region of constant depth, encounter an abrupt depth change, and then continue in another region of constant depth. Thus, depth will be piecewise constant
\begin{align}
\depth = 
\begin{cases}
\dup \quad \mbox{if } x<0 \\
\ddn \quad \mbox{if } x>0
\end{cases}
\end{align}
Most often, we will consider waves moving into shallower depth, so that $\dup > \ddn$. The incoming waves are randomized and generated with a peak forcing frequency of $\freqp$, which gives rise to peak wavelength $\lam = c/\freqp = \sqrt{g \depth} / \freqp$.


% Nondimensionalization
\subsection{Dimensionless variables}
\label{nondim}

% Table
%^^^^^^^^^^^^^^^^^^^^^^^^^^^^^^%
\begin{table}[h]%[htbp]
\begin{center}
\caption{Table of parameters}
\label{paramtable}
\begin{tabular}{l l l}
\hline \multicolumn{3} { c }{Parameters that are constant in a single experiment} \\
\hline Description & Notation and definition & Value in experiments \\
\hline
Peak forcing frequency		& $f_p$						& 2 Hz \\
Upstream depth			& $\dup$						& 12.5 cm \\
Downstream depth			& $\ddn$						& 3 cm \\
Characteristic wave amplitude	& $a = \mean{\eta^2}^{1/2} $		& 0.03--0.3 cm \\
Upstream wavelength		& $\lamup = \sqrt{g \dup}/f_p$		& 56 cm \\
Amplitude-to-depth ratio		& $\epsup = a / \dup$			& 0.0024--0.024 \\
Depth-to-wavelength ratio		& $\delup = \dup / \lamup$		& 0.22 \\
Depth ratio				& $\dratdn = \ddn/\dup$			& 0.24 \\
\hline \multicolumn{3} { c }{Parameters that change value across ADC} \\
\hline Description & Notation and definition \\
\hline
Local depth			& $\depth$			\\
Local wavespeed		& $c = \sqrt{g \depth}$	\\
Local peak wavelength	& $\lam = \sqrt{gd}/f_p$	\\
Local dimensionless depth		& $\drat = \depth/\dup$	
\end{tabular}
\end{center}
\end{table}
 %^^^^^^^^^^^^^^^^^^^^^^^^^^^^^^%
 
We introduce the following characteristic scales
\begin{align}
\label{ampdefn}
&\amp = \mean{\eta^2}^{1/2} 
&&\mbox{\em characteristic wave amplitude} \\
\label{lamdefn}
&\lam = \sqrt{g \depth} / \freqp
&&\mbox{\em characteristic wave length}
\end{align}
where $\mean{\cdot}$ indicates the mean of a quantity. Note that the characteristic wavelength $\lam = \lam(x)$ takes different values upstream and downstream of the ADC. We remark that experimental measurements indicate that the characteristic amplitude $\amp = \mean{\eta^2}^{1/2}$ is nearly the same on both values of the ADC. Hence, we will not distinguish between upstream and downstream values of $\amp$.
We introduce the following dimensionless variables
\begin{align}
&u = \eta/\amp
&&\mbox{\em dimensionless surface displacement} \\
&\tilde{x} = \frac{2 \pi \xi}{\lamfac \lam} = \frac{2 \pi (x-ct)}{\lamfac \lam}
&&\mbox{\em dimensionless position in moving frame} \\
& \tilde{t} = t \freqp / \lamfac
&&\mbox{\em dimensionless time}
\end{align}
We will consider the physical domain $\xi \in [0, \lamfac \lam]$, which is an integer multiple of the characteristic wavelength where $\lamfac$ will be selected later (probably about 5). Above, we have chosen the scale to normalize position as $\lamfac \lam/(2\pi)$ so that the dimensionless domain can be taken (after a shift) to be $\tilde{x} \in [-\pi,\pi]$. Accordingly, we have also chosen the time scale as $\lamfac/\freqp$, so that a wave traveling at the local wavespeed $c = \lam \freqp = \sqrt{g \depth}$ crosses the physical domain $x \in [0, \lamfac \lam]$ in one unit of dimensionless time. Said differently, the dimensionless wave speed is $(2 \pi)^{-1}$ so that a wave crosses the domain $[-\pi,\pi]$ in unit time.

%----------------------------------------------------------------------------%
% COMMENT ON LAMFAC
\begin{comment}
Old Note (around May 2019): After giving it some deliberation, I believe we should use $\lamfac=1$. My reason is that the peak in the wave spectrum occurs at a frequency of $\freqp$ or wavelength of $\lam$. In the tKdV Gibbs measure, the spectrum decays monotonically, so that the peak is at the lowest resolved frequency (or largest resolved wavelength). Thus, we want $\lam$ to correspond to the largest resolved wavelength, i.e.~the length of the periodic domain in the tKdV framework.

New Thought (July 2019): Actually, it is very possible that the bandwidth (2 Hz in the experiments) sets the value of $\lamfac$ that is most appropriate, since that bandwidth sets the decay rate of the spectrum. My thinking is that our experimental forcing does not really follow the upstream Gibbs measure all that well (because we did not have that in mind), but perhaps it best approximates the tail decay of some Gibbs measure. Perhaps, the lowest frequencies (i.e. those much slower than 2 Hz that are present in the theory but almost absent in the experiments) do not affect statistics that much.
\end{comment}
%----------------------------------------------------------------------------%

Dropping the tildes, the dimensionless KdV equation takes the form
\begin{equation}
\label{dimlessKdV}
u_t + {3 \pi} \epsup \drat^{-1} \, u u_x + \frac{4 \pi^3}{3 \lamfac^2} \delup^2 \drat \, u_{xxx} = 0
\qquad \text{for } x \in [-\pi,\pi]
\end{equation}
where dimensionless variables are as follows
\begin{align}
&\epsup = \amp/\dup \, , \\
&\delup = \dup/\lamup = \sqrt{\dup \freqp^2/g} \, \\
&\drat = {\depth}/{\dup} \, .
\end{align}
Here, $\drat$ is the local dimensionless depth, which changes value crossing the ADC. In particular, since \eqref{dimlessKdV} is in the reference frame that moves with the characteristic wave speed, we can assume that the ADC is met at $t = T_{ADC} = 0$. Then we have
\begin{equation}
\drat = 
\begin{cases}
1 		&\quad \mbox{for } t<0 \\
\dratdn 	&\quad \mbox{for } t>0
\end{cases}
\end{equation}
where $\dratdn = \ddn/\dup$.
See Table \ref{paramtable} for a summary of important parameters.

\begin{comment}
{\bf Sidenote}: In the case of constant depth, if one simply uses the most naive scales, $u = \eta/\amp$, $\tilde{x} = \xi/\lam$, $\tilde{t} = t \freqp$, then the dimensionless KdV is the more standard one:
\begin{equation}
2 u_t + 3 \eps u u_x + \frac{\delta^2}{3} u_{xxx} = 0
\end{equation}
where $\eps = a/h$ and $\delta = h/\lam$.
\end{comment}


\subsubsection{Alternative KdV formulation following Johnson}
Section \ref{nondim} makes the implicit assumption of continuous surface displacement crossing the ADC. This is a very physically motivated assumption, but there is a small amount of debate surrounding the assumption. An alternative is to appeal to Green's law and assume the quantity $\depth^{1/4} \eta$ to be continuous across the ADC, which necessarily implies a small discontinuity in the surface (which of course cannot be true literally), as is the approach taken in Johnson't book. The dimensional KdV equation would then be expressed as
\begin{equation}
2 (\depth^{1/4} \eta)_t + \frac{3 c}{\depth^{5/4}} (\depth^{1/4} \eta) (\depth^{1/4} \eta)_{\xi} + \frac{c \depth^2}{3} (\depth^{1/4} \eta)_{\xi \xi \xi} = 0
\end{equation}
Then, the dimensionless state variable is instead 
\begin{equation}
u = \frac{\depth^{1/4} \, \eta}{ \dup^{1/4} \, \amp}
\end{equation}
and the dimensionless KdV equation takes the form
\begin{equation}
u_t + \frac{3 \pi}{\lamfac} \epsup \drat^{-5/4} \, u u_x + \frac{4 \pi^3}{3 \lamfac^3} \delup^2 \drat \, u_{xxx} = 0
\qquad \text{for } x \in [-\pi,\pi]
\end{equation}
Notice the only difference is a small change in the power of $\drat$ in the second term (changes from $\drat^{-1}$ to $\drat^{-5/4}$).
We will not use this alternative formulation.




\subsection{Hamiltonian structure}

We now consider the Hamiltonian structure of the dimensionless variable-depth KdV equation \eqref{dimlessKdV}, over dimensionless domain $x \in [-\pi,\pi]$. First, the spatial average of any quantity is given by
\begin{equation}
\mean{q} = \frac{1}{2\pi} \int_{-\pi}^{\pi} q \dx \, .
\end{equation}
Appealing to near Ergodicity, we will typically assume equivalence between spatial, temporal, and ensemble averages. 

We measure the state variable $u$ as the displacement of the surface from equilibrium, so that $\mean{u} = 0$. Consequently, the momentum of $u$ vanishes
\begin{equation}
M[u] \defeq \int_{-\pi}^{\pi} u \dx = 0
\end{equation}
Next, the definition of the characteristic amplitude in \eqref{ampdefn} implies that $\mean{u^2} = 1$. Consequently, the energy of $u$ is fixed as
\begin{equation}
E[u] \defeq \frac{1}{2} \int_{-\pi}^{\pi} u^2 \dx = \pi
\end{equation}


We introduce the cubic and quadratic Hamiltonian components as
\begin{align}
H_3 = \frac{1}{6} \int_{-\pi}^{\pi} u^3 \dx	\, , \qquad
H_2 = \frac{1}{2} \int_{-\pi}^{\pi} u_x^2 \dx	\, .
\end{align}

Then the Hamiltonian of \eqref{dimlessKdV} is given by
\begin{equation}
\Ham = 3 \pi \epsup \drat^{-1} \, H_3 - \frac{4 \pi^3}{3 \lamfac^2} \delup^2 \drat \, H_2
\end{equation}
where $\drat$ changes value across the ADC. More specifically, there exists an upstream and a downstream Hamiltonian
\begin{align}
&\Hup = 3 \pi \epsup \, H_3 - \frac{4 \pi^3}{3 \lamfac^2} \delup^2 \, H_2 \\
&\Hdn = 3 \pi \epsup \dratdn^{-1} \, H_3 - \frac{4 \pi^3}{3 \lamfac^2} \delup^2 \dratdn \, H_2
\end{align}
$\Hup$ is a conserved quantity for $t<0$ and $\Hdn$ is a conserved quantity for $t>0$. Each Hamiltonian $\Ham^{\pm}$generates a corresponding Gibbs measure $\Gibbs^{\pm}$. (FILL IN DEFINITIONS)






\vsp{10} 
\subsection{Matching condition}

The statistical matching condition is
\begin{equation}
\meanup{\Hdn} = \meandn{\Hdn}
\end{equation}
An experimental observation is that the incoming skewness is small, thus
\begin{equation}
\meanup{H_3} \approx 0
\end{equation}
which immediately gives
\begin{equation}
\ampp D^{-3/2} \meandn{H_3} = \mu D^{1/2} \left( \meandn{H_2} - \meanup{H_2} \right)
\end{equation}
or equivalently
\begin{equation}
\label{H3H2}
\meandn{H_3} = \frac{\mu D^{2}}{\ampp} \, \Delta \mean{H_2}
\end{equation}
where $\Delta \mean{H_2} =  \meandn{H_2} - \meanup{H_2}$  indicates the difference between the upstream and downstream values.

To convert to dimensional, i.e.~experimental, values we use
\begin{align}
& \intt u^3 \dx = \frac{\mean{\eta^3}}{a^3} = 
\frac{\mean{\eta^3}}{\mean{\eta^2}^{3/2}} = \skw(\eta) \\
& \intt u_x^2 \dx = \left(\frac{\lambda_0}{a} \right)^2 \mean{\eta_x^2} 
= \left(\frac{\lambda_0}{a} \right)^2 \var(\eta_x)
\end{align}
Then \eqref{H3H2} gives the relationship
\begin{equation}
\frac{\skw(\eta)}{\Delta \var(\eta_x)} =
3 \mu D^2 \left( \frac{h_0}{a} \right)^3 = 3 \eps^{-3} \mu D^2
\end{equation}
In particular, the ratio on the left, which can be measured in the experiments, is predicted to scale as the inverse cube of the wave amplitude (as well as the square of the depth ratio).

%^^^^^^^^^^^^^^^^^^^^^^^^^^^^^^%
\begin{figure}%[htbp]
\begin{center}
\includegraphics[width = 0.80 \textwidth]{../Figs/SkewAmp.pdf}
\caption{\label{fig1} 
The ratio ${\skw(\eta)}/{\Delta \var(\eta_x)}$ plots against dimensionless wave amplitude.
}
\end{center}
\end{figure}
 %^^^^^^^^^^^^^^^^^^^^^^^^^^^^^^%


%----------------------------------------------------------%
% Comparison with experiments
%----------------------------------------------------------%
\section{Comparison with experiments}

\subsection{Comparison of basic features}

\subsection{New experimental measurements guided by theory}

%\bibliographystyle{plain}
%\bibliography{Notesbib}

\end{document}
